\documentclass{article}
\usepackage[utf8]{inputenc}
\usepackage[french]{babel}
\usepackage[T1]{fontenc}
\usepackage{tcolorbox}
\usepackage[margin=2cm]{geometry}
\usepackage{array}
\usepackage{hyperref}
\usepackage{tikz}
\usepackage{listings}
\usepackage{eurosym}
\usepackage{amsfonts}
\usepackage{amsmath}
\usepackage{cancel}

\title{IFT2015 automne 2016 - Devoir1} %trouver long trait d'union
\author{Philippe Caron}
\date{\today}

\setcounter{section}{2}
\renewcommand{\thesubsubsection}{\alph{subsubsection}.}
\renewcommand{\thefootnote}{\fnsymbol{footnote}}
\newenvironment{pseudo}{\begin{tcolorbox}[left skip = 2cm, right skip = 2cm]\itshape}{\end{tcolorbox}}
\newcommand{\key}[1]{{\bf #1}}
\newcommand{\name}[1]{{\scshape #1}}
\newcommand\tab[1][0.5cm]{\hspace*{#1}}

\lstset{frame=tb,
  language=PostScript,
  aboveskip=3mm,
  belowskip=3mm,
  showstringspaces=false,
  columns=flexible,
  basicstyle={\small\ttfamily},
  numbers=none,
  numberstyle=\tiny\color{pink},
  keywordstyle=\color{purple},
  commentstyle=\color{brown},
  stringstyle=\color{brown},
  breaklines=true,
  breakatwhitespace=true,
  tabsize=3
}

\begin{document}
\maketitle

\subsection{Qu'est-ce qu'il fait? [Postscript]}
\subsubsection{Manipulation de la pile}
Le contenu de la pile est le suivant à chaque opération (la valeur sur le dessus est représentée en bas):
$\stackrel{\text{État initial}}{
  \begin{tabular}{|c|}
    \hline 1 \\
    \hline 2 \\
    \hline 3.25 \\
    \hline 4e10 \\
    \hline -5 \\
    \hline 6 \\
    \hline 7 \\
    \hline 8 \\
    \hline
\end{tabular}}$
$\Rightarrow$
$\stackrel{\text{4 1 roll}}{
  \begin{tabular}{|c|}
    \hline 1 \\
    \hline 2 \\
    \hline 3.25 \\
    \hline 4e10 \\
    \hline 8 \\
    \hline -5 \\
    \hline 6 \\
    \hline 7 \\
    \hline
\end{tabular}}$
$\Rightarrow$
$\stackrel{\text{pop}}{
  \begin{tabular}{|c|}
    \hline 1 \\
    \hline 2 \\
    \hline 3.25 \\
    \hline 4e10 \\
    \hline 8 \\
    \hline -5 \\
    \hline 6 \\
    \hline
\end{tabular}}$
$\Rightarrow$
$\stackrel{\text{exch}}{
  \begin{tabular}{|c|}
    \hline 1 \\
    \hline 2 \\
    \hline 3.25 \\
    \hline 4e10 \\
    \hline 8 \\
    \hline 6 \\
    \hline -5 \\
    \hline
\end{tabular}}$
$\Rightarrow$
$\stackrel{\text{7 -4 roll}}{
  \begin{tabular}{|c|}
    \hline 4e10 \\
    \hline 8 \\
    \hline 6 \\
    \hline -5 \\
    \hline 1 \\
    \hline 2 \\
    \hline 3.25 \\
    \hline
\end{tabular}}$

\subsubsection{Opérateur mystique}
L'opérateur mystique sélectionne la plus grande valeur entre $a$ et $b$. Le contenu de la pile après son utilisation est donc $a$ si $a \geq b$ ou $b$ sinon.

\subsubsection{Parcours d'un tableau}
L'opérateur \key{tbl.?} compte le nombre d'éléments négatifs du tableau.

\subsubsection{Opérateur énigmatique}
L'opérateur énigmatique retourne le $n$-ième de la suite de Fibonacci, définie récursivement comme suit:
$$F_n =
\begin{cases}
  1 & \text{si } n \leq 1\\
  F_{n-1} + F_{n-2} & \text{sinon}
\end{cases}
$$

\subsubsection{Maximum dans un tableau}
La fonction suivante permet de calculer le maximum d'un tableau:
\begin{lstlisting}
  /tbl.max % [x0 x1 ..] tbl.max m
  {
    0 exch {
      2 copy lt { exch } if pop
    } forall
  } def
\end{lstlisting}
\end{document}
