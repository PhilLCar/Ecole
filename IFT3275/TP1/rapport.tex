\documentclass{article}

\usepackage[utf8]{inputenc}
\usepackage[french]{babel}
\usepackage{fontenc}
\usepackage[margin=1in]{geometry}
\usepackage{hyperref}
\usepackage{amsmath}
\usepackage{listings}

\title{Devoir 1 - IFT3275}
\date{\today}
\author{Philippe Caron}

\begin{document}
\maketitle
\section{Question 1}
\subsection{Déchiffrement}
Le message a pu être décodé en utilisant une attaque fréquentielle sur le texte chiffré. L'attaque a réévlé que le chiffre n'était pas une substitution monoalphabétique, mais plutôt un chiffre de vigenere avec un clef dont la longueur probable est 25.

Par la suite, il a suffit de trouver le décalage de la lettre la plus fréquente sur l'ensemble des positions $25 * i + k$ avec la lettre 'E', où $i$ vaut de $0$ à $L \div 25$, $L$ représente la longueur du document et $k$ vaut de $0$ à $25$. Par cette méthode on obtient la clef suivante:
\begin{center}
  LSJFHTFSLMBSVOHBOSVTTMBOE
\end{center}
Ce qui permet de déchiffrer le texte.
\subsection{Importance du texte}
Sans lire allemand, la recherche de la première phrase sur Google permet de trouver que ce texte est en fait un discours que Hitler a prononcé le 22 juin 1941. Une traduction est disponible ici: \url{http://research.calvin.edu/german-propaganda-archive/hitler4.htm}. Dans ce discours il annonce que l'Allemagne entrera en guerre contre la Russie, il parle également d'une «attaque au proportions jusqu'à lors inégalées» (traduction libre de ``At this moment, an attack unprecedented in the history of the world in its extent and size has begun''). \\
Sachant que l'Allemagne a attaquée l'Union Sovétique le 22 Juin, si ce message a effectivement été reçu le 19 Juin 1941, on ne peut vraiment pas dire qu'il soit sans importance! En fait, il donne la possibilité d'avertir les Russes de l'attaque imminente (ou de ne pas les avertir du tout). Il permet aussi de se positionner stratégiquement sachant qu'un fort contingent allemand risque de se diriger vers le front Ouest. Des machinations politiques à la stratégie militaire, ce message est une perle rare et ne devrait surtout pas être ignoré.

\section{Question 2}
Supposons que l'adversaire a accès au canal de transmission, par exemple, il a réussi à installer un dispositif qui sépare le signal du fil reliant la banque et le guichet de sorte qu'il peut recevoir une copie exacte des transmissions entre la banque et le guichet. Il ne connais peut-être pas exactement la manière dont l'information est codée, mais sa connaissance lui permet tout de même de séparé les message qui proviennent de la banque et ceux qui proviennent du guichet. Sinon il est fort probable que l'information soit transmise sur deux fils différents pour l'envoi et la réception, il n'a qu'à identifier ces fils, il peut également espionner le guichet et déduire que l'information qui survient immédiatement après une transaction vient du guichet, il pourrait éventuellement même installer un voltmètre dans son dispositif afin d'identifier le sens du courant.\\
Tout cela n'a guère d'importance cependant, l'important est qu'il est possible pour l'adversaire de connaitre exactement l'information qui est envoyée du guichet à la banque pour un temps donné et vice-versa, que ce soit par un canal secondaire ou non. Si le flux n'utilise par d'IV, il lui suffit de copier le comportement du guichet pour une transaction passée. Sans même comprendre ce qu'il fait, il va pouvoir recevoir de la banque le même résultat qu'à la transaction précédente, exemple: un retrait de 20\$. Puis il peut répéter en volant des transactions jusqu'à vider le guichet. En volant des petits montants, 20\$ par personne par exemple, il y de bonne chances que ni la banque, ni les clients ne se rende compte de la supercherie avant très longtemps.\\
Il est toutefois très probable que le flux soit initialisé aléatoirement à chaque fois, sans quoi la sécurité est serait vraiment très réduite. À ce moment, il aurait l'option de recueillir les message envoyés par le gichet et tenter de comprendre le fonctionnement du système en effectuant la même transaction à répétition. Au nombre de retrait de 20\$ qui sont faits à chaque jour, il ne risque pas de prendre trop de temps avant d'accumuler un répertoire de textes chiffrés assez considérable. Plus haut, on parlait de mimer les répones du guichet, mais on peut aussi mimer le comportement de la banque, c'est d'ailleurs favorable pour un coup rapide car aucune limite de 200\$ par jour n'est applicable si la banque n'est pas au courant des retrait. À chaque transaction le guichet demande une autorisation et la banque doit répondre par une forme de «OUI» ou «NON», si cette réponse n'est pas trop longue, l'adversaire peu faire une attaque exhaustive sur le guichet. Il peut rajouter des bits au messages, essayer de déclencher des erreurs en envoyant des messages aléatoires.

\section{Question 3}
Le message $m$ envoyé par Alice est authentifié par RSA comme suit:
\begin{equation}
  D(m)=m^d \pmod{N}
\end{equation}
Pour le décrypter, Bob qui possède $e$ et $N$ fait:
\begin{equation}
  E(m)=m^e \pmod{N}
\end{equation}
A priori, Bob n'a que récupéré le message, mais ce message contient plus d'information qu'il n'en parait. Dans le cas qui nous intéresse, $PGCD(m, N) \neq 1$. Or on sait que $N = pq$ et que $p$ et $q$ sont premier. Donc forcément, disons que $m = p$, $N = mq$. Quand bob recevra le message authentifié par Alice, il aura $m$ en sa possession, puisqu'il a déjà $N$, il peut déduire $q$ en faisant $q = N \div m$. Il aura alors $p$ et $q$, à ce moment, déduire d sera trivial puisque:
\begin{equation}
  d = e\pmod{(p-1)(q-1)}^{-1}
\end{equation}
Bob aura alors en sa possession les clefs publique et privée $((N, e), d)$ ce qui permet à Bob de lire tout ce qu'Alice lit, et de signer des message en son nom sans qu'il soit possible pour les autres de les différencier. Ce cas précis constitue évidemment une faille du système, mais est-ce vraiment une faiblesse dans la sécurité de RSA?\\
Si l'on considère qu'Alice ne fait que transmettre des informations à Bob par rapport à un sujet quelconque, par exemple la hauteur de la neige dehors. Elle ne vérifie jamais la valeur de $m$ par rapport a $p$ et $q$ puisque cette comparaison n'a aucun intérêt pour elle. Il est alors possible que $m = p$, puisque la valeur de $m$ est déterminé par quelque chose d'aléatoire (dans ce cas-ci la hauteur de la neige, mais n'importe quel transmission qui n'est pas organisée risque de comporter une forte composante aléatoire) alors Bob a autant de chances de trouver $p$ en regardant les valeurs de $m$ qu'en devinant des nombres premiers au hasard. Au final, cette faiblesse a tellement peu de chance de se produire qu'elle n'en est pas vraiment une.\\
Elle pourrait être comparée à la probabilité qu'une clef XOR ne soit composée que de 0, ce qui pour un générateur totalement aléatoire est équiprobable à n'importe quelle autre clef. Malgré cette éventualité, la probabilité est si faible que ce n'est pas réellement considérer comme une faiblesse.

\pagebreak
\section{Question 4}
Voici la clef RSA générée:
\hspace{1.5cm}
\begin{lstlisting}
-------------------------- RSA PUBLIC KEY ----------------------------
N=
6126753213598620157697723900316460032761171889742582442524140921488008
1545564306041096838501924032824388219712718612181625172423192019695503
9985311555932583881720602673863352028711467553487138174726830506936353
8859557647343283159373314470658143717320037554745654036987665432681420
7173687663120601482792020804805553678529419014173310952899646009845105
7351721000026878872426808336056280811537806670289020520370182906728092
0412907030341357078409255361986757381049435813699305158662941419831548
0468207336886524789430052432766690991420658538076890504049370996597419
852606773381801784969741820337494191913284734438961831879

e=
683929

------------------------ RSA PRIVATE KEY -----------------------------
d=
2088346835213212962753164131203054404415297367534598334280477622529717
9049382380558380441807915053141608311032093080289042902765695518124622
9990913969932720822884349598476409615087640294539387312964749530196872
3224220347234816562361627911711256255648001173860743608658515103682296
3282569484950354456631779574619524014415892056178544915504552888835195
1917018451581624068085686272517410654881800525778769459076809818513552
1766828649813487592806789029264571488007919614013350913955377928890175
2463875407294436738004453904240313631265293644662906662197424258618782
271728976262378137923784784090542653905491300549706316137

======================================================================
p=
5080980958023975098409839481049810394801937401934091327501938409130497
1023947013977401937409137409137409137409173409173049173049712093740197
4031913904701937403197409137409137409137409137409173409198095720982043
9237957487509247509475072909020139093175091173059170571039471039571009
8460938290850298024895097345878572740582783572087203872970301380297527
8906967

q=
1205820935802395203957230957230957023975023975093275029375029375023975
0293750293750392750239750239750239750239750293750293754029384023984029
3570297513920859052703598402937508982309372809273283572093857023252242
321870582930492374079328957204890375084252093752337
\end{lstlisting}

\pagebreak
\section{Question 5}
Si nous n'avions que la clef $k_1 \in K$ à déduire, il suffirait de faire une attaque exhaustive en essayant toutes les clefs, du genre $$ \text{DECODE}_{k_1}(c)\stackrel{?}{=}m $$
Cette attaque nécessite d'évaluer DECODE un maximum de $|K|$ fois, avec une espérance de $\frac{|K|}{2}$. Mais on ne possède pas $\text{CODE}_{k_1}(m)$, rendant impossible cette approche. Cependant ont peut mettre cette expression en relation avec la seconde clef:
\begin{align*}
  c = \text{CODE}_{k_2}(\text{CODE}_{k_1}(m))\\
  \text{DECODE}_{k_2}(c) = \text{CODE}_{k_1}(m)
\end{align*}
À ce moment, si on suppose que pour tout $x$, $\text{DECODE}_{k_x}(c)$ retourne une valeur différente, on peut essayer successivement toutes les clefs $k_1$ de l'ensemble $K$ pour chacune des clefs $k_2$ du même ensemble, ce qui revient à élavuer CODE $|K|^2$ fois, et DECODE $|K|$ fois (ou l'inverse dépendamment de quelle clef on met dans la boucle) donc $|K|^2+|K|$ évaluations au total. Puisqu'on a présumé qu'il n'y avait pas de recoupement, au moment ou $\text{DECODE}_{k_2}(c) = \text{CODE}_{k_1}(m)$ on sait que $k_1$ et $k_2$ ont les bonnes valeurs. Pour accélérer le processus, si cela vaut la peine, on peut d'emblée rejeter les éléments qui ne sont ni dans $M$ ni dans $C$, puisque par définition $\text{DECODE}_{k_2}(c) = \text{CODE}_{k_1}(m) \in (M \cap C)$.

\end{document}
