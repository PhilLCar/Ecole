\documentclass{article}
\usepackage[utf8]{inputenc}
\usepackage[french]{babel}
\usepackage[T1]{fontenc}
\usepackage{tcolorbox}
\usepackage[margin=2cm]{geometry}
\usepackage{array}
\usepackage{hyperref}
\usepackage{tikz}
\usepackage{listings}
\usepackage{eurosym}
\usepackage{amssymb}
\usepackage{amsmath}
\usepackage{cancel}
\usepackage{siunitx}
\usepackage{textcomp}

\title{\vspace{7cm}IFT2125 automne 2016 - Devoir 3} %trouver long trait d'union
\author{Philippe Caron}
\date{\today}

%\setcounter{section}{0}
\renewcommand{\thesubsubsection}{(\alph{subsubsection})}
\renewcommand{\thefootnote}{\fnsymbol{footnote}}
\newenvironment{pseudo}{\begin{tcolorbox}[left skip = 0.5cm, right skip = 0.5cm]}{\end{tcolorbox}}
\newcommand{\key}[1]{{\bf #1}}
\newcommand{\name}[1]{{\scshape #1}}
\newcommand\tab[1][0.5cm]{\hspace*{#1}}

\lstset{frame=tb,
  language=Python,
  aboveskip=3mm,
  belowskip=3mm,
  showstringspaces=false,
  columns=flexible,
  basicstyle={\small\ttfamily},
  numbers=left,
  numberstyle=\tiny\color{gray},
  keywordstyle=\color{purple},
  commentstyle=\color{blue},
  stringstyle=\color{brown},
  breaklines=true,
  breakatwhitespace=true,
  tabsize=3
}

\begin{document}
\maketitle

\pagebreak

\section{Mergesort 5}
Commençons par la fonction \key{merge5()}. Avant la boucle, l'ajout des sentinelles et la détermination de la longueur sont faits en temps constant, puis dans la boucle, le programme fait toujours 5 comparaison avant d'avoir déterminer le bon cas, où il fait deux opération, disons donc environ 7 opération \key{nlen} fois (voir code), où \key{nlen} correspond à la somme des longueur de chaque tableau. Soit \key{n} le nombre d'élément par tableau, on peut donc dire que:
$$t_{merge5}(n)= k + 5n \cdot 7 = k + 35n \in O(n)$$
La fonction \key{mergesort5()} quant à elle est également constante quand $\key{n} \leq 1$, sinon lorsqu'elle divise la liste en sous liste, on peut imaginer que chaque séparation est $O(n)$ ainsi que chaque appel à \key{merge5()}, on a donc 10 appel de $O(n)$ ce qui est aussi dans $O(n)$. Il ne nous reste qu'à savoir combien de fois \key{mergesort5()} est appelée. Ceci correspond essentiellement au temps que cela prend pour que le paramètre \key{n} vaille 0. Puisqu'on divise en 5 à chaque fois, on trouve le nombre avec $\log_5n$ et donc on sait que l'algorithme est $O(n \log n)$.
$$t_{mergesort5}(n)=\log_5n \cdot O(n) \in O(n\log n)$$

\section{Roule}
\subsubsection{arbre d'appel et fonction}
\begin{lstlisting}
roule(a, 143)
	roule(a, 136)
		roule(a, 128)
			roule(a, 8)
				roule(a, 1) = a
				roule(a, 2) = a^2
			roule(a^2, 16)
				roule(a^2, 4)
					roule(a^2, 2) = a^4
					roule(a^4, 2) = a^8
				roule(a^8, 4)
					roule(a^8, 2) = a^16
					roule(a^16, 2) = a^32
		roule(a, 8)
			roule(a, 1) = a
			roule(a, 2) = a^2
	[a^32 * a^2 = a^34]
	roule(a, 7)
		roule(a, 4)
			roule(a, 2) = a^2
			roule(a^2, 2) = a^4
		roule(a, 3)
			roule(a, 2) = a^2
			roule(a, 1) = a
		[a^2 * a = a^3]
	[a^4 * a^3 = a^7]
[a^34 * a^7 = a^41]
\end{lstlisting}
On conclu donc que $roule(a, 143) = a^{41}$
\subsubsection{récurrence}
Le nombre d'appel peut-être donné par la fonction suivante:
\begin{equation}
  t(k)=
  \begin{cases}
    1 \hspace{1cm} & \text{si } k = 0\\
    2 \cdot t(k - 1) & \text{sinon}\\
  \end{cases}
\end{equation}
\subsubsection{fonction}
On peut définir $t(k)$ explicitement:
\begin{equation}
  t(k) = 2^k
\end{equation}
Il est donc certain que si on dit $f(n)=2^n$, alors $t(k) \in \Theta f(n)$.

\section{RSA}
On connait les valeurs suivantes:
\begin{align*}
  p &= 311\\
  q &= 47\\
  z &= 14617\\
  n &= 13\\
  m &= 123\\
\end{align*}
On trouve $\phi$:
$$ \phi = (p - 1)(q - 1) = 310 \times 46 = 14260 $$
puis on trouve $s$ avec une simple boucle en python:
\begin{lstlisting}
>>> for x in range(14617):
...     if (13 * x) % 14260 == 1:
...             print(x)
... 
1097
\end{lstlisting}
ensuite on calcule $c$:
$$ c = m^n \mod z = 123^{13} \mod 14617 = 1925 $$
puis on confirme:
$$ c^s \mod z = 1925^{1097} \mod 14617 = 123 $$

Le message est restauré correctement les valeurs trouvées sont donc les bonnes.

\pagebreak

\section{Distance minimale}
Voici l'algorithme:
\begin{pseudo}
  \name{FindMin}$(P, n)$:\\
  \tab{}$min\leftarrow\infty$\\
  \tab{}\key{for} $i=0$ \key{to} $n$:\\
  \tab{}\tab{}\key{for} $j=i+1$ \key{to} $n$:\\
  \tab{}\tab{}\tab{}\key{if} \{distance entre $P[i]$ et $P[j]$\} $< min$ \key{then}:\\
  \tab{}\tab{}\tab{}\tab{}$min\leftarrow $ \{distance entre $P[i]$ et $P[j]$\}\\
  \tab{}\key{return} $min$\\

  \name{CenterMin}$(P, d)$:\\
  \tab{}\{trie selon $y$ le tableau de points $P$\}\\
  \tab{}$min\leftarrow d$\\
  \tab{}\key{for} $i=0$ \key{to} $len(P)$:\\
  \tab{}\tab{}\key{for} $j=i+1$ \key{until} $len(P)$ \key{or until} \{distance entre $P[i]$ et $P[j]$\} $ < min$:\\
  \tab{}\tab{}\tab{}\key{if} \{distance entre $P[i]$ et $P[j]$\} $< min$ \key{then}:\\
  \tab{}\tab{}\tab{}\tab{}$min\leftarrow $ \{distance entre $P[i]$ et $P[j]$\}\\
  \tab{}\key{return} $min$\\
  
  \name{MinDistRec}$(P, n)$:\\
  \tab{}\key{if} $n \leq 3$ \key{then}:\\
  \tab{}\tab{}\key{return} \name{FindMin}$(P, n)$\\
  \tab{}\key{else}:\\
  \tab{}\tab{}$P_1 \leftarrow P[0:n/2]$\\
  \tab{}\tab{}$P_2 \leftarrow P[n/2:n]$\\
  \tab{}\tab{}$d_1 =$ \name{MinDistRec}$(P_1, n / 2)$\\
  \tab{}\tab{}$d_2 =$ \name{MinDistRec}$(P_2, n - n / 2)$\\
  \tab{}\tab{}$d = min(d_1, d_2)$\\
  \tab{}\tab{}\{conserve les points de $P$ à l'intérieur d'une distance ± $d$ du centre: $(P[n/2-1] + P[n/2]) / 2$\}\\
  \tab{}\tab{}\key{return} $min(d,$ \name{CenterMin}$(P[-d : +d], d)$\\
  
  \name{MinDist}$(P)$:\\
  \tab{}\{trie selon $x$ le tableau de points $P$\}\\
  \tab{}\key{return} \name{MinDistRec}$(P, len(P))$\\
\end{pseudo}
On peut voir que cet algorithme est $\notin \Omega(n^2)$ en analysant chacune des fonctions séparément.
D'abord la première (\name{FindMin}) est $O(n^2)$, mais elle n'est jamais utilisée avec un $n > 3$ on peut donc considérer qu'elle est effectuée en temps constant.
La seconde (\name{CenterMin}) est très similaire à \name{FindMin} alors on pourrait croire qu'elle est aussi $O(n^2)$ or c'est faux puisqu'il est prouvé que la seconde boucle ne dépasse jamais 6 itérations.Le trie est donc l'opération la plus demandante avec $O(n\log n)$.
Similairement, la fonction principale demande un trie elle aussi que l'on peut considérer $O(n\log n)$.
L'ordre de l'algorithme sera donc déterminé par le nombre de fois que \name{MinDistRec} est exécutée, et puisqu'on divise par deux à chaque fois, on sait que c'est $O(\log n)$.

L'algorithme final est donc $O(n (\log n)^2)$, ce qui est plus petit que $O(n^2)$.

\pagebreak
\begin{lstlisting}
import math

def merge5(S, T, U, V, W):
	R = list()
	s = t = u = v = w = 0
	nlen = len(S) + len(T) + len(U) + len(V) + len(W)
	S.append(float("inf"))
	T.append(float("inf"))
	U.append(float("inf"))
	V.append(float("inf"))
	W.append(float("inf"))
	for x in range(nlen):
		if S[s] < T[t]:
			if S[s] < U[u]:
				if S[s] < V[v]:
					if S[s] < W[w]:
						R.append(S[s])
						s += 1
					else:
						R.append(W[w])
						w += 1
				else:
					if V[v] < W[w]:
						R.append(V[v])
						v += 1
					else:
						R.append(W[w])
						w += 1
			else:
				if U[u] < V[v]:
					if U[u] < W[w]:
						R.append(U[u])
						u += 1
					else:
						R.append(W[w])
						w += 1
				else:
					if V[v] < W[w]:
						R.append(V[v])
						v += 1
					else:
						R.append(W[w])
						w += 1
		else:
			if T[t] < U[u]:
				if T[t] < V[v]:
					if T[t] < W[w]:
						R.append(T[t])
						t += 1
					else:
						R.append(W[w])
						w += 1
				else:
					if V[v] < W[w]:
						R.append(V[v])
						v += 1
					else:
						R.append(W[w])
						w += 1
			else:
				if U[u] < V[v]:
					if U[u] < W[w]:
						R.append(U[u])
						u += 1
					else:
						R.append(W[w])
						w += 1
				else:
					if V[v] < W[w]:
						R.append(V[v])
						v += 1
					else:
						R.append(W[w])
						w += 1
	return R
		

def mergesort5(A, n):
	if n <= 1:
		return A
	else:
		p = 0
		gap = math.ceil(n / 5)
		B = A[p : p + gap] ; p += gap
		C = A[p : p + gap] ; p += gap
		D = A[p : p + gap] ; p += gap
		E = A[p : p + gap] ; p += gap
		F = A[p : p + gap]
		return merge5(
			mergesort5(B, len(B)),
			mergesort5(C, len(C)),
			mergesort5(D, len(D)),
			mergesort5(E, len(E)),
			mergesort5(F, len(F)))


# TEST
a = [11, 3, 4, 7, 2, 9, 10, 8, 1, 5, 6]
mergesort5(a, len(a));
\end{lstlisting}
\end{document}
