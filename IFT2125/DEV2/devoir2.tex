\documentclass{article}
\usepackage[utf8]{inputenc}
\usepackage[french]{babel}
\usepackage[T1]{fontenc}
\usepackage{tcolorbox}
\usepackage[margin=2cm]{geometry}
\usepackage{array}
\usepackage{hyperref}
\usepackage{tikz}
\usepackage{listings}
\usepackage{eurosym}
\usepackage{amssymb}
\usepackage{amsmath}
\usepackage{cancel}
\usepackage{siunitx}
\usepackage{textcomp}

\title{IFT2125 automne 2016 - Devoir 2} %trouver long trait d'union
\author{Philippe Caron}
\date{\today}

%\setcounter{section}{0}
\renewcommand{\thesubsubsection}{(\alph{subsubsection})}
\renewcommand{\thefootnote}{\fnsymbol{footnote}}
\newenvironment{pseudo}{\begin{tcolorbox}[left skip = 2cm, right skip = 2cm]\itshape}{\end{tcolorbox}}
\newcommand{\key}[1]{{\bf #1}}
\newcommand{\name}[1]{{\scshape #1}}
\newcommand\tab[1][0.5cm]{\hspace*{#1}}

\lstset{frame=tb,
  language=Python,
  aboveskip=3mm,
  belowskip=3mm,
  showstringspaces=false,
  columns=flexible,
  basicstyle={\small\ttfamily},
  numbers=left,
  numberstyle=\tiny\color{gray},
  keywordstyle=\color{purple},
  commentstyle=\color{blue},
  stringstyle=\color{brown},
  breaklines=true,
  breakatwhitespace=true,
  tabsize=3
}

\begin{document}
\maketitle
\section{Récurrence}
Soit la récurrence suivante:
$$
t_n =
\begin{cases}
  3 & \text{si } n = 0 \text{,}\\
  3 & \text{si } n = 1 \text{,}\\
  17 & \text{si } n = 2 \text{,}\\
  at_{n-1} + bt_{n-2} + ct_{n-3} + (d + en)2^n & \text{si } n > 2 \text{,}
\end{cases}
$$
dont la solution est:
\begin{equation}
  t_n = (2n - 3)5^n + (6 - 2n^2)2^n
\end{equation}
On cherche $a, b, c, d, e$.

Tout d'abord, en examinant la solution de la récurrence en (1), on peut d'emblée affirmer que les zéros du polynôme associé sont 2 et 5, et qu'ils sont de multiplicité 3 et 2 respectivement. Ce qui nous donne un polynôme de la forme suivante:
\begin{equation}
  (r-2)^3(r-5)^2
\end{equation}

Ceci nous donne suffisamment d'information pour trouver les coefficiant du polynôme de gauche lorsqu'on isole les termes qui ne dépendent pas d'un $t_n$ précédent\footnote{Sachant que l'annihilateur d'un terme de la forme $k^xP(y)$ est $(E-x)^{y+1}$}:
\begin{align*}
  t_n &= at_{n-1} + bt_{n-2} + ct_{n-3} + (d + en)2^n\\
  t_n - at_{n-1} - bt_{n-2} - ct_{n-3} &= (d + en)2^n\\
  \Rightarrow (r^3 - ar^2 - br - c) \cdot (r-2)^2
  &\stackrel{(2)}{=} (r-2)^3(r-5)^2\\
  r^3 - ar^2 - br - c &= (r-2)(r-5)^2\\
  r^3 - ar^2 - br - c &= r^3 - 12r^2 + 45r -50\\
  \Rightarrow a &= 12\\
  b &= -45\\
  c &= 50
\end{align*}

Ensuite il nous suffit de résoudre un simple système 2 équations 2 inconnues en utilisants les prochaines valeurs de $t_n$ calculées grâce à l'équation (1).
\begin{equation}
  \begin{aligned}
    t_3=279 &= at_{n-1} + bt_{n-2} + ct_{n-3} + (d + en)2^n\\
    &= 12 \cdot 17 - 45 \cdot 3 + 50 \cdot 3 + 8d + 24e\\
    60 &= 8d + 24e\\
    15 &= 2d + 6e
  \end{aligned}
\end{equation}
\begin{equation}
  \begin{aligned}
    t_4=2709 &= at_{n-1} + bt_{n-2} + ct_{n-3} + (d + en)2^n\\
    &= 12 \cdot 279 - 45 \cdot 17 + 50 \cdot 3 + 16d + 64e\\
    -24 &= 16d + 64e\\
    -3 &= 2d + 8e
  \end{aligned}
\end{equation}

La solution au système formé des équations (3) et (4) est la suivante:
\begin{align*}
  \Rightarrow d = \frac{69}{2}\\
  e = -9
\end{align*}

Ce qui termine la recherche des coefficients.

\pagebreak

\section{Dénominations}

Soit les dénominations $1 = d_0, d_1, ...,d_k$ où $5d_{i-1} \leq d_i$. Il faut prouver que la procédure vorace trouvera toujours un résultat optimal.

Prenons un montant $m$, le cas dangereux est celui où il existe un multiple de petites dénominations (appelons la $d_p$) plus petit que le nombre de $d_{p-1}$ qu'il faudra pour combler l'espace entre $m$ et $d_{p+1}$ et ou $d_p$ et $d_{p+1}$ sont premiers entre eux.

\paragraph{Construisons une monaie hypotéthique autour de ce cas.}

Commençons avec une pièce de 1\$ ($d_{p+1}=100$). Nous cherchons le plus gros $d_p$ possible, donc:
$$\max d_p\text{ | }5d_p \leq d_{p+1} \Rightarrow d_p = 20$$
or 20 divise 100. La plus petite pièce qui respecte les condition est donc celle de 19\textcent. Choisissons maintenant $d_{p-1}$. Afin de maximiser les chances d'échec, nous prendrons la pièce de 1\textcent.

Soit la dénomination suivante:
$$d_0 = 1$$
$$d_1 = 19$$
$$d_2 = 100$$

Vérifions que la dénomination ci-haute respecte les conditions initiales:

\begin{align*}
  5 \cdot d_2 = 5 \leq d_1 = 19\\
  5 \cdot d_1 = 95 \leq d_2 = 100
\end{align*}

\paragraph{Échec de l'algorithme vorace.}

Si l'on soummet $1.14\$$ à l'algorithme vorace, celui-ci retournera $1 \times 1\$ + 14 \times 1$\textcent, pour un total de 15 pièces. Or la solution optimale aurait été de remettre 6 pièces de 19\textcent (considérablement moins).

La preuve est donc faite qu'un système de dénominations qui respecte la condition initiale ne peut pas nécessairement être traité par un algorithme vorace.

\pagebreak

\section{Code de Huffman}
Pour encoder la chaîne suivante :\texttt{lepessimismeoooooooodeoooooooodonald}, on fait un tableau des fréquence des lettres:

\begin{center}
  \begin{tabular}{c|c}
    lettre & fréquence \\
    \hline \hline
    o & 17 \\
    e & 4 \\
    s & 3 \\
    d & 3 \\
    l & 2 \\
    i & 2 \\
    m & 2 \\
    p & 1 \\
    n & 1 \\
    a & 1
  \end{tabular}
\end{center}

À partir de ce tabeau on se crée un arbre:
\vspace{2cm}

\begin{tikzpicture}[
    scale = 1.5, transform shape, thick,
    every node/.style = {draw, circle, minimum size = 8mm},
    grow = down,
    level 1/.style = {sibling distance = 5cm},
    level 2/.style = {sibling distance = 6.5cm},
    level 3/.style = {sibling distance = 3.5cm},
    level 4/.style = {sibling distance = 1.8cm},
    level 5/.style = {sibling distance = 1.5cm},
    level 6/.style = {sibling distance = 1cm},
    level distance = 1.25cm
  ]

  \node (Start) {36}
  child { node (A) {17} }
  child {
    node (B) {19}
    child {
      node (C) {8}
      child { node (D) {4} }
      child {
        node (E) {4}
        child { node (F) {2} }
        child { node (G) {2} }
      }
    }
    child {
      node (H) {11}
      child {
        node (I) {5}
        child { node (J) {2} }
        child {
          node (K) {3}
          child { node (L) {1} }
          child {
            node (M) {2}
            child { node (N) {1} }
            child { node (O) {1} }
          }
        }
      }
      child {
        node (P) {6}
        child { node (Q) {3} }
        child { node (R) {3} }
      }
    }
  };
  
  \begin{scope}[nodes = {draw = none}]
\textbf{
    \node at (A) [below = 2mm] {O};
    \node at (D) [below = 2mm] {E};
    \node at (F) [below = 2mm] {M};
    \node at (G) [below = 2mm] {I};
    \node at (J) [below = 2mm] {L};
    \node at (L) [below = 2mm] {P};
    \node at (N) [below = 2mm] {N};
    \node at (O) [below = 2mm] {A};
    \node at (Q) [below = 2mm] {S};
    \node at (R) [below = 2mm] {D};
    }
    \path (Start) -- (A) node [near start, left]  {0};
    \path (Start) -- (B) node [near start, right] {1};
    \path (B)     -- (C) node [near start, left]  {0};
    \path (C)     -- (D) node [near start, left]  {0};
    \path (C)     -- (E) node [near start, right] {1};
    \path (E)     -- (F) node [near start, left]  {0};
    \path (E)     -- (G) node [near start, right] {1};
    \path (B)     -- (H) node [near start, right] {1};
    \path (H)     -- (I) node [near start, left]  {0};
    \path (I)     -- (J) node [near start, left]  {0};
    \path (I)     -- (K) node [near start, right] {1};
    \path (K)     -- (L) node [near start, left]  {0};
    \path (K)     -- (M) node [near start, right] {1};
    \path (M)     -- (N) node [near start, left]  {0};
    \path (M)     -- (O) node [near start, right] {1};
    \path (H)     -- (P) node [near start, right] {1};
    \path (P)     -- (Q) node [near start, left]  {0};
    \path (P)     -- (R) node [near start, right] {1};
  \end{scope}
\end{tikzpicture}

\pagebreak

Avec quoi on peut faire une table qui fait correspondre chaque lettre à son encodage de Huffman:
\begin{center}
  \begin{tabular}{c|l|c}
    lettre & code  & longueur\\
    \hline \hline
    o & 0      & 1\\
    e & 100    & 3\\
    s & 1110   & 4\\
    d & 1111   & 4\\
    l & 1100   & 4\\
    i & 1011   & 4\\
    m & 1010   & 4\\
    p & 11010  & 5\\
    n & 110110 & 6\\
    a & 110111 & 6
  \end{tabular}
\end{center}
La longueur de Huffman du mot est donc:
$$ 17 \times 1 + 4 \times 3 + 3 \times 4 + 3 \times 4 + 2 \times 4 + 2 \times 4 + 2 \times 4 + 5 + 6 + 6 = 94$$
La suite la plus courte est évidemment la suite qui maximise le nombre de «o», comme par exemple la suite \texttt{anpmildseooooooooooooooooooooooooooo} (de longueur 67).

\pagebreak

\section{Algorithme}
\subsubsection{}
Le couple $(\{1, 2, ..., n\}, \{S \subseteq {1, 2, ..., n} : \sum\limits_{i \in S} \omega_i \leq \omega_0\})$ n'est pas un matroïde, car l'ensemble $I$ ne respecte pas la propriété de l'échange.

Imaginons l'ensemble $A=\{i, x, y\} \subset \{1, 2, ..., n\} : \omega_x + \omega_y = \omega_i$, et l'ensemble $B=\{i, u, v, w\} \subset \{1, 2, ..., n\} : \omega_u + \omega_v + \omega_w = \omega_i$. Par définition, $A$ et $B$ sont éléments de $I$.

\paragraph{Preuve} Selon le principe de l'échange, $[A \in I$ et $B \in I$ et $|A| < |B|] \Rightarrow (\exists a \in B \backslash A)[X \cup \{a\} \in I]$. Autrement dit, puisque $A$ est plus court que $B$, il y a un élément $a$ de $B$ qui peut être ajouté à $A$ de sorte que $A \cup \{a\} \in I$. Or c'est impossible puisque tout ajout non nul à $A$ ferait augmenter le poids au dessus de $\omega_i$. Ce qui signifie que l'échange n'est pas toujours possible et donc qu'on ne parle pas ici d'une matroïde.
\paragraph{Exemple} $A=\{15, 7, 8\}$ et $B=\{15, 3, 4, 8\}$, peut importe l'élément de $B$ choisi, $A \notin I$.
\subsubsection{}
De \key{(a)} on peut déduire qu'il existe parfois plusieurs façons différentes d'arriver au même poids, et donc que la méthode vorace risque de ne pas être optimale dans tous les cas.
\subsubsection{}
Voici les valeurs obtenues par l'algorithme:
\begin{center}
\begin{tabular}{|c|c|}
  tuple & Maximum de $\sum\limits_{i=1}^nx_i\omega_i$ trouvé sur cet exemplaire \\
  \hline \texttt{p1} & $\stackrel{\hspace{9cm}}{14}$\\
  \texttt{p2} & 132\\
  \texttt{p3} & 158\\
  \texttt{p4} & 1496\\
  \texttt{p5} & 1799\\\hline
\end{tabular}
\end{center}
\subsubsection{}
Voici le code en python (commentaires enlevés):
\begin{lstlisting}
    def swapsort(t):
        l = list(t)
        for x in range(0, len(l) - 1):
            for y in range(1, len(l) - x):
                if l[y - 1] > l[y]:
                    l[y - 1], l[y] = l[y], l[y - 1]
        return tuple(l)

    def Subsetsum(p):
        w0 = p[0]
        wi = 0;
        p = swapsort(p[1:])
        p = p[::-1]
        for x in range(len(p)):
            while(w0 > wi + p[x]):
                wi += p[x]
        return wi
\end{lstlisting}
\end{document}
